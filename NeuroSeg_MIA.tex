\documentclass[times,twocolumn,final]{elsarticle}

%% Stylefile to load MEDIMA template
\usepackage{medima}
\usepackage{framed,multirow}

%% The amssymb package provides various useful mathematical symbols
\usepackage{amssymb}
\usepackage{latexsym}

% Following three lines are needed for this document.
% If you are not loading colors or url, then these are
% not required.
\usepackage{url}
\usepackage{xcolor}

\usepackage{hyperref}

%%%%%
\usepackage{amsmath}
\usepackage[numbers]{natbib}
\usepackage{siunitx}
\usepackage{booktabs}
\usepackage{multirow}
%%%%%
\definecolor{newcolor}{rgb}{.8,.349,.1}

\newcommand{\xj}[1]{\textcolor[rgb]{1,0,0}{(CXJ:#1)}}
\newcommand{\zj}[1]{\textcolor[rgb]{1,0,1}{(ZJ:#1)}}
\newcommand{\md}[1]{\textcolor[rgb]{0,0,1}{#1}}
\newcommand{\td}[1]{\textcolor[rgb]{0.5,0.0,0.5}{#1}}
\newcommand{\delete}[1]{}

\journal{Medical Image Analysis}

\begin{document}
	
\verso{Given-name Surname \textit{et~al.}}

\begin{frontmatter}

\title{Progressive Learning for Large-scale Neuronal Population Reconstruction}                      
\tnotetext[tnote1]{This is an example for title footnote coding.}

\author[1]{Given-name1 \snm{Surname1}\corref{cor1}}
\cortext[cor1]{Corresponding author: 
	Tel.: +0-000-000-0000;  
	fax: +0-000-000-0000;}
\author[1]{Given-name2 \snm{Surname2}\fnref{fn1}}
\fntext[fn1]{This is author footnote for second author.}
\author[2]{Given-name3 \snm{Surname3}}
%% Third author's email
\ead{author3@author.com}
\author[2]{Given-name4 \snm{Surname4}}

\address[1]{Affiliation 1, Address, City and Postal Code, Country}
\address[2]{Affiliation 2, Address, City and Postal Code, Country}


\begin{abstract}
Reconstruction of neuronal populations from large-scale optical microscopy (OM) brain images is essential to investigate neuronal circuits and brain mechanisms.
To mitigate the impact of low quality of images on neuron reconstruction, some studies have been conducted to extract neuron voxels using deep neural networks (DNNs).
However, training such a DNN usually relies on a large number of images with voxel-wise annotations, which is very costly to obtain in terms of both finance and labor.
In this paper, we propose an \md{unsupervised} progressive learning framework for robust neuron reconstruction free of manual annotations. Our framework consists of a neuron tracing module and a segmentation network, which mutually complement and progressively promote each other.
Based on the progressively learned model, reconstructing neurons from local noisy image blocks is feasible with allowed memory.
To reconstruct neuronal populations from a terabyte-sized large-scale image, we introduce a new automatic reconstruction algorithm by adaptively tracing neurons block by block and assembling individual neurons continuously and smoothly.

We build a dataset called ``VISoR-40" that consists of 40 OM image blocks from mouse cortical regions. Extensive experimental results on our VISoR-40 dataset and the public BigNeuron dataset demonstrate the effectiveness and superiority of our method for neuronal population reconstruction and single neuron reconstruction, respectively.
Furthermore, we successfully apply our method to the reconstruction of dense neuronal populations in a large-scale mouse brain slice.
\end{abstract}

\begin{keyword}
\KWD Deep learning \sep Image segmentation\sep Neuronal population reconstruction\sep Large-scale images\sep Optical microscopy
\end{keyword}

\end{frontmatter}


\section{Introduction}
\label{sec:introduction}

%Reconstruction of neuronal morphology is important for brain studies. 
A blue print of the brain architecture, including the morphology and interconnectivity of neuronal populations, allows for measuring and visualizing neuronal structure, understanding neuronal identity, and determining potential connectivity.
Therefore, complete reconstruction of neuronal populations from ultra-scale brain images is essential to investigate the mechanism of the nervous system, analyze brain changes, and facilitate our understanding of brain diseases such as dementia and Alzheimer's disease~\cite{Petrella2003, Giorgio2013}.
One of the key techniques in this endeavor is optical microscopy (OM), which allows detailed visualization of neurons and makes the reconstruction of every neuron possible~\cite{Senft2011}, as shown in Fig.~\ref{fig:brain}.
However, despite numerous efforts devoted, this task is still one of the main challenges in computational neuroscience.

\begin{figure}[t]
	\centering
	\includegraphics[width=1\columnwidth]{./Illustrations/brain2.pdf}
	\caption{A 3D OM image for a mouse brain slice captured by the VISoR imaging system~\cite{Wang2019}. The large density of neurons, low contrast, image noises, and huge volume pose significant challenges for automatic reconstruction of neuronal populations from the ultra-scale image.}	
	\label{fig:brain}
\end{figure}

The challenges in large-scale neuronal population reconstruction are mainly caused by complex morphology of neurons, low quality, and huge volume of OM brain images.
Due to the complicated process of imaging acquisition and the uneven distribution of fluorescence markers in neurons, the voxel intensities vary dramatically in highly noisy and inhomogeneous environments.
Moreover, to detail the structure of neurons in the brain, OM images in high resolution are required. Such an image typically contains trillions of voxels. It is even impractical to directly load the entire image into a computer memory before reconstructing neurons.
In the other hand, manual reconstruction of neuronal populations from ultra-scale brain images is an extremely laborious and time-consuming task as well.
Moreover, sophisticated knowledge of neuron morphology is also required for manual reconstruction.
%
Therefore, an effective and automatic algorithm for reconstructing large-scale neuronal populations in these challenging situations is greatly desired in practice.
 % 
 
 
Attempts have been made for neuron reconstruction from large-scale OM images in recent years, such as Neuron Crawler~\cite{Zhou2015}, UltraTracer~\cite{Peng2017} and MEIT~\cite{Wang2018}.
A common solution is to divide the large-scale image into blocks and then trace neurons block by block.
Despite their great improvements in this task, several limitations still remain.
%
One main bottleneck is that the base tracers used for neuron tracing in image blocks typically employ a series of traditional image processing algorithms such as binarization, fast marching, ray-shooting, etc. 
Unfortunately, these algorithms using hand-crafted features and rules have difficulty in reconstructing neurons from low-contrast and noisy image blocks.
To improve the reconstruction quality, deep learning techniques have been adopted in neuron reconstruction recently~\cite{Xu2016, Li2017, Zhou2018, Kozinski-MIA2020}. 
However, these approaches require huge amount of manually annotated data to train the deep neuron segmentation networks.
%

Compared with manual annotations, the reconstruction results by conventional algorithms effectively provide approximate locations of neurons. 
Although these voxels may not cover all neurons precisely, they provide important cues for obtaining complicated patterns of neurons.
Therefore, we propose a progressive learning scheme for neuronal population reconstruction (PLNPR) to take advantage of both conventional methods and deep learning techniques.
More specifically, we employ conventional methods to produce pseudo-labels for training a deep neural network for neuron segmentation. 
The network is expected to learn more comprehensive features of neurons from noisy labels. 
With a more powerful DNN for segmenting neurons, the neuron reconstruction using conventional tracers could be improved. 
Then we progressively refine the segmentation DNN with better neuron reconstruction results as pseudo labels and reconstruct more complete neurons with better neuron segmentation.
We first investigate this concept in our preliminary work \cite{Zhao2019} on a few image blocks. 
In this work, we apply PLNPR on dense neuron population reconstruction from an ultra-scale OM image. 

Another limitation of existing large-scale neuron reconstruction methods~\cite{Zhou2015, Peng2017, Wang2018} is that they mainly focus on single neuron reconstruction. 
%
For dense neuronal population reconstruction from ultra-scale images, dense neurites may cross with each other, making the neuron tracing much more challenging in low-contrast and noisy OM images. 
% 
Following the commonly used block-by-block framework, we introduce UltraNPR, which utilizes our PLNPR approach as the base tracer in blocks, and design novel tracing and fusion strategies for reconstructing dense neuronal populations.
%
We start local reconstruction using our PLNPR in the blocks where soma can be detected, and propagate a group of neurite tips as pseudo somas for neighboring blocks to trace the neuronal population. 
%
There are over-tracing and topological discrepancy in the reconstructed neurites in adjacent blocks. 
We design a fusion method with spatially varying confidences in order eliminate the tracing errors that usually occur at boundary regions of local blocks.   


In summary, we propose PLNPR, a progressive learning framework that integrates traditional tracing methods and deep segmentation networks for neuron population reconstruction without using manual annotations.
%
Then we introduce UltraNPR by integrating PLNPR with adaptive block-wise propagation and fusion strategies which can reconstruct dense neuronal populations from trillions of voxels in ultra-scale OM images. 
%
In order to evaluate our method, we build a dataset ``VISoR-40" which consists of 40 OM image blocks from mouse cortical regions. Manual annotations of eight blocks in the dataset are available for the community. Extensive experiments on our VISoR-40 dataset and the BigNeuron dataset demonstrate the effectiveness and superiority of our progressive learning algorithm for both neuronal population reconstruction and single neuron reconstruction.
The reconstructed neuron population from an ultra-scale image of a mouse brain slice shows the robustness of our UltraNPR.




%In this work, we introduce UltraNPR, an algorithm designed for reconstructing dense neuronal populations from large-scale or even ultra-large-scale images. UltraNPR reconstructs neuronal populations by progressively improving the completeness of neuronal structure block by block.
%
\delete{Firstly, the large-scale raw image is divided into blocks of the same size.
Since somas are where signals from the dendrites are joined and pass on, UltraNPR begins the reconstruction from the blocks that contain somas.
% which can be detected using existing soma detection methods.
For robust reconstruction from low-quality images, our PLNPR method is applied to trace neurons in each block.
Based on the reconstruction results in already-reconstructed blocks, UltraNPR automatically and adaptively searches the next block to trace.
%For each block containing somas, this reconstruction procedure repeats until no new terminal tips could be detected or the next block contains somas.
%Finally, UltraNPR analyzes the reconstruction results in all blocks, and assembles the matched neurites from adjacent blocks to obtain the final complete neuronal population reconstruction. In our implementation, 
Finally, a fusion algorithm is designed to make the fragmented neurites of a neuron from adjacent blocks can be assembled continuously and smoothly.
In this way, UltraNPR is capable of exploring a large-scale image for neuronal population reconstruction.



%%%%%%%%%%%%%%%%%%%%%%%%%%%%%%%%%%%%%%%
In summary, we make the following contributions in this work. 
\begin{enumerate}
\item We propose a general progressive learning framework that can integrate existing neuron tracing and deep segmentation networks for neuron reconstruction without using manual annotations.

\item We integrate our PLNPR with a block-wise propagation and fusion strategy which can reconstruct dense neuronal populations from trillions of voxels in an image slice. 

\item We build a dataset ``VISoR-40" which consists of 40 OM image blocks from mouse cortical regions. Manual annotations of eight blocks in the dataset are available. 
%which consists of 40 OM images from mouse cortical regions and contains about 400 neuronal trees. Publishing this dataset will strongly support the study of deep neural networks for microscopic explorations.
\item Extensive experiments on our VISoR-40 dataset and the public BigNeuron dataset demonstrate the effectiveness and superiority of our progressive learning algorithm for both neuronal population reconstruction and single neuron reconstruction.
\end{enumerate}
}
 
\section{Related Work}
\label{sec:related work}

\subsection{Techniques for Robust Neuron Reconstruction}
\label{sec:neuron reconstruction}
Early conventional techniques for neuron reconstruction typically employ traditional image processing algorithms, such as snakes~\cite{Wang2011, Cai2006}, principal curves~\cite{Bas2011}, graph theory~\cite{Peng2010a, Yang2013, De2016}, model-fitting~\cite{Zhao2011, Santamaria2015}, watershed~\cite{Navlakha2013, Suembuel2016}, energy minimization~\cite{Quan2013, Liu2016}, mean-shift clustering~\cite{Frasconi2014}, ray-shooting~\cite{Wu2014, Liu2019}, fast-marching~\cite{Peng2011, Xiao2013, Liu2018} and so on.
Unfortunately, all these conventional algorithms rely on hand-crafted features and carefully tuned parameters, and usually tend to fail when the image quality is poor.
%TReMAP~\cite{Zhou2016}, NGPST~\cite{Quan2015},

To improve the reconstruction performance from low-quality image blocks, some machine learning based methods were introduced to extract neuron voxels for neuron reconstruction. This kind of methods employs various classifiers with hand-crafted features, such as support vector machine (SVM)~\cite{Chen2015}, minimum spanning tree~\cite{Basu2016}, Bayesian probabilistic model~\cite{Radojevic2017}, Bootstrap aggregating~\cite{Wang2017}, gradient boosting decision trees (GBDT)~\cite{Gu2017}, Markov chain Monte Carlo (MCMC)~\cite{Skibbe2015, Skibbe2019} and so on.
However, the main limitation of these methods is that hand-crafted features usually suffer from limited representation capability for accurate recognition, considering more challenging and complex image blocks.


Recently, we have evidenced an increasing development of deep learning based methods~\cite{Li2017, Zhou2018, Xu2016}, which bring the power of DNNs to improve the reconstruction performance. Instead of manually designing sophisticated features, these methods learn feature representations in a data-driven way and extract more distinctive features. When more complicated classifiers are employed to segment neuron voxels from image blocks, these methods can achieve more robust reconstruction results. 
Though great improvement of neuron reconstruction could be achieved, these deep learning based methods rely on strong supervision for network training, i.e., manual annotations for neuron voxels.
Unfortunately, due to the complicated morphology of neurons and the low quality of OM images, such annotations are very costly to obtain in terms of both time and labor.
%Though great improvement of neuron reconstruction could be achieved on a specific dataset, these deep learning based methods rely on strong supervision for network training, i.e., manual annotations for dense voxels. Unfortunately, due to the dense distribution of neuronal populations in the OM images, such annotations are extremely time-consuming and require expensive labor to obtain.
%
In comparison, we propose a novel iterative framework to progressively improve the 3D DNN-based neuron reconstruction performance without using manual annotations.


\subsection{Large-scale Neuron Reconstruction}
\label{sec:largescale}

Most existing neuron reconstruction methods focus on robust and accuracy neuron reconstruction from local noisy image blocks. Despite substantial advancements in these methods, they often need to load all image voxels into memory before the reconstruction and the sheer volume of a large-scale image is far beyond the processing capability for them, especially on the memory cost and tracing time.
%
In recent years, some attempts have been made to reconstruct neurons from large-scale OM images, such as Neuron Crawler~\cite{Zhou2015}, UltraTracer~\cite{Peng2017} and MEIT~\cite{Wang2018}.
To tackle the challenges caused by the large volume of images, a common solution is to reconstruct neuronal morphology block by block. Each block is cropped from the raw image and is much smaller in size than the raw image.
Therefore, existing tracing methods, such as APP2~\cite{Xiao2013}, MOST~\cite{Wu2014} and FMST~\cite{Yang2018}, can be directly used as the base tracer in their methods to trace neurites in each block. Then the reconstructed neurites in all blocks are assembled to obtain the final reconstruction.


However, all of these methods focus on single neuron reconstruction and the images they process usually contain only one single neuron.
%Despite their great improvement in neuron reconstruction from large-scale images, all of them mainly focus on single neuron reconstruction and the images they process usually contain only one neuron.
%However, a large-scale OM brain image usually contains dense neuronal populations in practice.
Since these methods are mainly designed for large-scale single neuron reconstruction, they are not useful for dense neuronal populations, in which neurons frequently contact or close to each other in the image.
%These methods can be extended to sparsely spaced neurons but are not useful for dense neuronal populations, in which neurites frequently contact or close to each other in the image. 
%If closely spaced neurites belonging to different neurons are not distinguished, individual neurons may be assembled in the reconstruction. 
When the closely spaced neurites that belong to different neurons are not be distinguished, neurons can not be separately reconstructed.
Therefore, a complete reconstruction of dense neuronal populations from large-scale images still remains challenging for existing methods.
%
In this work, we introduce a new algorithm to reconstruct neuronal populations from large-scale or even ultra-large-scale images.

\section{Proposed Method}
\label{sec:method}

In this section, we describe the details of our methods, including PLNPR and UltraNPR. Given a noisy and large-scale OM brain image, PLNPR aims to obtain a robust and accurate reconstruction of neurons from local noisy image blocks, while UltraNPR aims to reconstruct complete neuronal populations from a large-scale image more efficiently.
%In the reminder of this section, we first introduce the PLNPR algorithm, then we detail the UltraNPR algorithm.
%The pipeline of our methods is shown in Fig.~\ref{fig:framework}.
%In the reminder of this section, we first introduce the PLNPR, a unsupervised progressive learning algorithm which can significantly improve neuronal population reconstruction performance without using manual annotations. Then we detail the UltraNPR algorithm for reconstructing neuronal population from a ultra-scale 3D mouse brain slice.


\subsection{PLNPR for Robust Neuronal Population Reconstruction}
%\subsection{Unsupervised Progressive Learning for Neuronal Population Reconstruction}
\label{sec:PLNPR}

To reconstruct neuronal populations from noisy OM images, our PLNPR algorithm consists of three key components: a segmentation network, an image enhancement module and a neuron tracing module, as shown in Fig.~\ref{fig:framework}. 
%
The segmentation network is designed to extract neuron voxels from noisy and complex backgrounds.
%\de{To reconstruct neurons from noisy OM images, our PLNPR method consists of three key components: a segmentation network, image enhancement module, and a neuron tracing module, as shown in Fig.~\ref{fig:framework}. The segmentation network is designed to first extract neuron voxels from noisy and complex background.}
%
Then, instead of using a binary mask of the segmented voxels, we employ a probability map predicted by the network and integrate it with the raw image intensities in order to simultaneously preserve the global neuron structure and local neurite details.
After that, the neuron tracing module is applied to reconstruct neurons from the enhanced image block.
%However, existing segmentation networks are trained with strong supervision, and dense voxel-wise annotations are required. Instead of acquiring annotations with huge human efforts, 
We progressively train the segmentation network with the reconstructed neurons inferred from conventional tracing methods as labels, and improve the reconstruction results from better neuron segmentation.
%\de{Instead of acquiring manual annotations with huge efforts, we progressively train the segmentation network with the reconstructed neurons as labels using conventional neuron tracing methods,}
Based on the iterative learning process, the powerful DNNs and tracing methods mutually complement and promote each other to gradually improve the neuron reconstruction performance.


\subsubsection{Progressive Learning without Annotations}
\label{sec:PL}

%In order to learn discriminative and representative features for extracting neuron voxels, we train the segmentation network with the reconstructed neurons as the 
In order to train the segmentation network to learn discriminative features for extracting neuron voxels, we use the reconstructed neurons to provide pseudo labels.
%
In each iteration of segmentation and reconstruction, we apply the NGPST~\cite{Quan2015} as the neuron tracing module to reconstruct neurons from image blocks. This module can be replaced by any tracing method that does not require manual annotations for training.
%\de{we utilize the NGPST~\cite{Quan2015} to reconstruct a neuronal population in the OM image. The neuron tracing module can be replaced by any tracing method that does not require voxel-wise annotations for training.}
It takes an image block $\mathbf{B}$ in size of $S\times H \times W$ as input and reconstructs a neuronal population with separated neurons.
%The intensity $\mathbf{I}(x)$ of a voxel $x$ belongs to $[0,1]$.
From the reconstruction results, we produce a binary mask $\mathbf{M}$ indicating foreground by $\mathbf{M}(x)=1$ and background by $\mathbf{M}(x)=0$ for a voxel $ x $.



Given $N$ image blocks, we train our segmentation network using the neuron masks $\{\mathbf{M}_i, i=1,\cdots,N\}$ generated from the neuron tracing module.
Details of the network architecture and training strategy are described in Sec.~\ref{sec:network}.
The output of the neuron segmentation network is a 3D probability map $\mathbf{P}$, which is computed by a voxel-wise softmax activation function. $\mathbf{P}(x)\in [0,1]$ indicating the probability of a voxel $x$ to be a neuron part.
Then, by fusing the predicted probability map with the raw intensities, the raw block is further enhanced in order to preserve both local signals and global structures simultaneously. Details are introduced in Sec.~\ref{sec:enhancement}.
When the enhanced block is used as input to the neuron tracing module, more complete neuronal populations can be reconstructed.
%\de{Then the raw image is further enhanced by fusing the predicted probability map with the raw intensities in order to preserve both global structures and local signals simultaneously.Details are introduced in Sec.~\ref{sec:enhancement}. %here the weak neuron signals will be enhanced, as show in Fig. \ref{fig:framework}. By feeding the enhanced image to the neuron tracing module, more complete neuronal populations can be reconstructed.}

\subsubsection{DNN for 3D Neuron Segmentation}
\label{sec:network}

Considering the size, morphology and intensity of neurons vary significantly, extracting neuron voxels from image blocks is not a trivial problem.
In recent years, many 3D DNNs, such as 3D U-Net~\cite{Cicek2016}, 3D DSN~\cite{Dou2017} and DenseVoxNet~\cite{Yu2017} have demonstrated an outstanding capability in various biological and biomedical image segmentation tasks.
Therefore, we take advantage of 3D segmentation networks to extract more representative features to meet the challenges of neuron segmentation.
In this work, the 3D DSN is extended as our neuron segmentation network to balance the performance and computation burden.
%\de{Since neurons vary significantly in size, morphology, and intensity, we take the advantage of 3D segmentation networks to extract more representative features to meet the challenges of neuron segmentation from 3D OM images. The network can be any end-to-end trainable 3D segmentation network, such as 3D U-Net~\cite{Cicek2016}, 3D DSN~\cite{Dou2017} and DenseVoxNet~\cite{Yu2017}. In this paper, we extend the 3D DSN as our neuron segmentation network to balance the performance and computation burden.} 
%In addition, to further validate the robustness and effectiveness of our algorithm, we also test several different networks in our framework and the results are introduced in Sec.~\ref{sec:experiments}.


The 3D DSN network consists of convolutional layers, pooling layers and deconvolutional layers, all in 3D fashion. The convolutional layers and pooling layers act as feature extractor, while the deconvolutional layers followed by softmax layer aim to up-sample the feature maps to the same size as the input. To further boost the information flow within the network, two more branches are employed to connect the shallower layers to the output layer. These connections strengthen the gradient propagation, stabilize the learning process, and further taps the potential of the limited training data to learn more discriminative features. %Please refer to~\cite{Dou2017} for more details.


Although 3D DSN has achieved excellent performance for 3D organ segmentation~\cite{Dou2017}, it is still prone to overfitting in our case due to the limited training data.
%\de{Although DSN has demonstrated its excellent performance in volumetric medical image segmentation, it is still prone to overfitting in our case due to limited training data.}
One common solution to this problem is to combine the predictions of several DNN models at the test time. However, this approach will greatly reduce the efficiency of the system.
Instead of training several models simultaneously, we employ a more efficient alternative solution for network training, which is known as Dropout~\cite{Srivastava2014}.
This technique can significantly reduce node interactions and help the network to learn more robust features that better generalize to new data.
In our network, the dropout with a rate of $0.5$ is applied to every convolutional layer.
%In our network, the dropout layer is implemented following each convolutional layer with the dropout rate of $0.5$.
%\de{We employ the dropout technique for our network training and the dropout layer is implemented following each convolutional layer with the dropout rate of $0.5$.}


Another challenge of training 3D DNN is the memory limitation because the 3D feature images are huge with respect to the input size. Therefore, for each input image block, we crop a group of cubes in size of $160\times 160\times 160$ with $30\%$ overlaps, and set batch size to 1 during training. Correspondingly, at the test time, we stitch these overlapped probability cubes together using average blending to get a probability map in the same size with the input block.


In addition, the volume of neuron (foreground) voxels is usually much smaller than that of background in an OM image.
To cope with this imbalance problem, a data balancing technique is introduced for network training.
%This highly imbalance problem could suppress the performance of the segmentation network. To cope with this problem, a data balancing technique is introduced for network training.
Specifically, when computing the training loss, we only consider the neuron voxels and a certain portion of background voxels, which is randomly selected as non-neuron samples.
The number of non-neuron voxels used for training is set as $10$ times that of neuron voxels.
%\de{Moreover, the volume of non-neuron voxels is typically much larger than that of neurons in images. This data imbalance could suppress the performance of the neuron segmentation network. We use a data balancing technique to address the problem. Specifically, for each cube, we randomly select a certain portion of background voxels as non-neuron samples, while the rest background voxels are not included in the loss computation. The number of background voxels for training is set as $10$ times that of neuron voxels.}

Last but not least, to have the same physical resolution with the lateral dimension in OM image blocks, voxels along the axial dimension are interpolated after the imaging process. However, it makes the image quality along different dimensions inhomogeneous. In order to alleviate the impact of this problem on network training, a random transposition process is employed for each cube as data augmentation.
%\de{Last but not the least, voxel intensities along the axial dimension in OM images are interpolated to the same resolution with the lateral dimension after the imaging process. Due to the inhomogeneity of imaging quality along different dimensions, a random transposition process of each cube is employed as data augmentation for training.}

%To train the networks, we use the stochastic gradient descent algorithm with the Adam update rule as the stochastic optimization strategy and a categorical cross-entropy loss function. We use a decaying learning rate starting from 1?−4 and gradually decreased to 1?−7 on the last epoch. We use an early stopping policy by monitoring validation performance and picked the best model with the highest accuracy on the validation set.

\subsubsection{Image Enhancement}
\label{sec:enhancement}

After finishing the network training, we use the trained model to predict the probability cube for each input cube.
Then, by stitching all probability cubes together, we can obtain the probability map $\mathbf{P}$ for the entire raw image block $ \mathbf{B} $.
Each element in $\mathbf{P}$ indicates the probability that the corresponding voxel in $ \mathbf{B} $ belongs to the neuron.
To utilize the probability map, one natural way is to reconstruct neurons directly from it.
However, since the pseudo labels is not as accurate as manual annotations, the performance of the segmentation network is limited. As a result, the predicted probability map may not cover all neurons and some local details may be lost, especially for the first few iterations.
In our method, an enhanced representation is generated by fusing the probability map and original raw image block, in order to keep detailed structures and suppress noise signals effectively.
Specifically, a new probability map $\widetilde{\mathbf{P}} $ is first constructed by linearly mapping the value range of $ \mathbf{P}\in [0,1] $ to the value range $[\mathbf{B}_{min}, \mathbf{B}_{max}]$ of $\mathbf{B}$.
%\de{By using the trained segmentation network, a probability cube of the corresponding input cube is predicted. Then all probability cubes are stitched together to obtain a probability map $\mathbf{P}$ for the entire image stack. Each element of $\mathbf{P}$ indicates the probability of the corresponding voxel as a neuron, known as foreground probability. A natural way to use the probability map is to apply the tracing algorithm directly on it to reconstruct neurons. However, due to the limited performance of the DNN network that is trained with pseudo labels, the probability map may not cover all neurons and lose some signal details compared to the raw image $ \mathbf{I}$, especially at the first few iterations. In our approach, we fuse the original raw image and the probability map together to get an enhanced representation, in order to suppress noise signals and keep detailed structures effectively. Specifically, we construct a new probability map $\widetilde{\mathbf{P}} $ by linearly mapping the  value range of $ \mathbf{P}\in [0,1] $ to the value range $[\mathbf{I}_{min}, \mathbf{I}_{max}]$ of the raw image $\mathbf{I}$.}
\begin{equation}
\widetilde{\mathbf{P}}(x) = (\mathbf{B}_{max}-\mathbf{B}_{min})\mathbf{P}(x).
\end{equation}


Then, base on the raw image block $\mathbf{B}$ and the probability map $\widetilde{\mathbf{P}}$, an enhanced image block $\mathbf{E}$ is computed as
\begin{equation}
\mathbf{E}(x) = \alpha\widetilde{\mathbf{P}}(x) + (1-\alpha)\mathbf{B}(x),
\label{equ: enhance}
\end{equation}
where $\alpha\in [0,1]$ is a weight to control the contributions of voxel $ x $ in the original intensity and the probability map. By feeding the enhanced blocks to the neuron tracing module, neuronal populations can be reconstructed more completely.
%\de{Then we use a straightforward way to compute an enhanced image $\mathbf{A}$ from the probability map $\widetilde{\mathbf{P}}$ and the raw image $\mathbf{I}$ as where $\alpha$ is a weight to control the contributions of voxel $ x $ in the probability map and the original intensity. When the enhanced images are used as input to the neuron tracing module, morce complete neuronal populations can be reconstructed.}


With more reliable reconstruction results for supervision, the segmentation network could be further trained to learn more discriminative and representative features for producing probability maps, which in turn benefits the tracing module to reconstruct neurons in the next iteration.
%Our system progressively improves the neuron reconstruction performance by combining conventional tracing methods and DNNs without any manual annotations.
\begin{figure}[t]
	\centering
	\includegraphics[width=1\columnwidth]{./Illustrations/ngps.pdf}
	\caption{
		Our progressive learning technique gradually improves the segmentation network to extract neuron signals from (a) raw image which has noises and low contrast. (b)(c)(d)(e) The probability maps generated by the segmentation network at different iterations. (f) Combing the probability map and the raw intensity, the enhanced block preserves both global trajectory and local details. (g)(h)(i)(j) More and more complete and accurate reconstruction of the neuronal populations can be obtained with more iterations. (k) The manually labeled neurons are shown for comparison. Separated neurons are shown in different colors.}
	\label{fig:ngps}
\end{figure}
%
As shown in Fig.~\ref{fig:ngps}(a), due to the noises and low contrast in the raw image block, the intensity of neuron voxels is inhomogeneous, which makes some neurons subtle.
%At first, by feeding the raw image to the conventional method~\cite{Quan2015}, a neuronal population can be reconstructed. However, compared to the ground truth (GT) shown in Fig.~\ref{fig:ngps}(k), the reconstructed neuronal population is incomplete and many neurites are missing, as Fig.~\ref{fig:ngps}(g) shows.
At first, comparing with the ground truth (GT) shown in Fig.~\ref{fig:ngps}(k), the neurons reconstructed from the raw image block are incomplete and many neurites are missing, as shown in Fig.~\ref{fig:ngps}(g).
Then, by utilizing the pseudo labels derived from imperfect reconstruction, the segmentation network can be trained to learn features for global trajectories. Fig.~\ref{fig:ngps}(b) shows the predicted probability map, which demonstrates the enhanced trajectories.
With more iterations of neuron reconstruction and network training, more distinctive and long-range trajectory features can be progressively captured by the network, as shown in Fig.~\ref{fig:ngps}(c)(d)(e).
By combining the original image intensities with the predicted probability map, both local signal details and global trajectories are well preserved in the enhanced block, as Fig.~\ref{fig:ngps}(f) shows.
%In order to preserve global trajectories and local signal details well, we combine the original image intensities with the predicted probability map. The enhanced image is shown in Fig.~\ref{fig:ngps}(f).
Iteration by iteration, the completeness and accuracy of neuron reconstruction are increasingly improved, as shown in Fig.~\ref{fig:ngps}(h)(i)(j).
%the neuronal populations are reconstructed more and more completely and accurately, as shown in Fig.~\ref{fig:ngps}(h)(i)(j).
%\de{As shown in Fig.~\ref{fig:ngps}(a), neurites are subtle because of the low contrast and noises in the raw image. At the beginning, the reconstructed neurons are incomplete and many neurites are missing, as Fig.~\ref{fig:ngps}(g) shows, compared to the ground truth (GT) shown in Fig.~\ref{fig:ngps}(k). Then trained by the pseudo labels derived from non-perfect reconstruction, the segmentation network is able to learn features for global trajectories, and the predicted probability maps demonstrate the enhanced trajectories, as Fig.~\ref{fig:ngps}(b) shows. With more iterations of network training and neuron reconstruction, our segmentation network captures more distinctive and long-range trajectory features progressively, as shown in Fig.~\ref{fig:ngps}(c)(d)(e). By combining the probability map with the raw image intensities, both global trajectories and local signal details are well preserved. Fig.~\ref{fig:ngps}(f) shows the enhanced image. Iteration by iteration, the reconstruction of neuronal populations becomes more and more complete and accurate, as shown in Fig.~\ref{fig:ngps}(h)(i)(j).}









\subsection{Large-scale Neuronal Population Reconstruction}
\label{sec:UltraNPR}

\begin{figure*}[th]
	\centering
	\includegraphics[width=1\textwidth]{./Illustrations/framework_ultranpr.PNG}
	\caption{Diagram of our UltraNPR algorithm for neuronal population reconstruction in a large-scale brain slice.}
	\label{fig:ultra_framework}
\end{figure*}

To reconstruct neuronal populations from a large-scale 3D image, our UltraNPR consist of four key components: a soma detection module, a local reconstruction module, a block search module, and a neurites fusion module, as shown is~\ref{fig:ultra_framework}.
At first,  the large-scale image is divided into overlapped 3D blocks of the same size. 
The soma detection module is used to detect somas from the large-scale image.
The local reconstruction module is based on our PLNPR algorithm to reconstruction neurons from low-quality image blocks. 
Then, to reconstruct the large-scale image block by block,  we repeatedly use the block search module to determine what the next block needs to be reconstructed. 
After that, the neurites fusion module is applied to obtain a complete and continuous reconstruction from adjacent blocks.

\subsubsection{Initial Soma Detection}
\label{sec:soma}

To detect somas from the large-scale image efficiently, we apply soma detection algorithm~\cite{Quan2013} on each block separately.
However, due to the overlap between blocks, somas in the overlapped area would be detected repeatedly.
To tackle this over-detection error, we map all detected somas from local blocks to the coordinate system of the original image and merge the overlapped somas by averaging their position and radius.
The soma detection results from the large-scale image are visualized in Fig.~{4} of the supplementary file.
These somas are also used to define the initial blocks for neuron reconstruction in the next step.


%\subsubsection{Large-scale Neuronal Population Reconstruction}
\subsubsection{Block Search Policy and Local Reconstruction}
\label{sec:trace}

The neuronal population reconstruction $\mathbf{R}$ in a large-scale image can be modeled as the joint performance of the reconstruction in each block $\mathbf{B}_i$.
Given an image $\mathbf{I}$, performing a local tracing method $T$ to reconstruct neuronal populations block by block can be formulated as follows
\begin{equation}
\mathbf{R} = T(\mathbf{I}) = T(\mathbf{B}_1, \mathbf{B}_2,\cdots,\mathbf{B}_N).
\end{equation}

As somas are where signals from the dendrites are joined and pass on, the blocks containing somas are the most probable locations to start the tracing.
Therefore, the initial blocks that contain somas are firstly reconstructed by our PLNPR method.
The reconstruction produced by PLNPR is represented by a series of neuronal compartments. The neuronal compartments which are close to the block boundary typically indicate the continuity of the structure of neurons. We call these compartments as terminal tips and use them as the potential continuous signals for searching the next blocks to grow the neuronal structure from each initial block.
%
Then, neurons in the next blocks are reconstructed by PLNPR based on the terminal tips provided by adjacent blocks that have already been reconstructed.
By iteratively searching the next blocks, UltraNPR can reconstruct neuronal populations more and more completely.

%
In addition, due to the dense distribution of neurons in the brain image, fragmented neurites belonging to different neurons could be in the same block, which means some blocks may be repeatedly reconstructed starting from different initial blocks.
To more efficiently explore the structure of neuronal populations, for each searching direction of an initial block, the iterative searching process will be terminated when the next block contains somas or no new terminal tips could be detected.


\subsubsection{Neurites Fusion from Adjacent Blocks}
Since neurons would be split into fragmented neurites when dividing the raw image into blocks, neurites from adjacent blocks need to be maximally connected, and the connection should be as continuous and smooth as possible.
As fragment neurites in the overlapping area may belong to different neurons, we first match them by comparing the overlapping region between every two neurites from adjacent blocks.
Then, the matched neurites are assembled to get a continuous and smooth reconstruction. 

However, simply assembling them would cause over-tracing error and topological discrepancy, as shown in Fig.~\ref{fig:overlap_discrepancy}.
The over-tracing error is caused by the overlap between blocks when the respective neurites from these blocks are assembled.
The topological discrepancy is mainly caused by the lack of context information for tracing methods, which makes the reconstruction near the block boundary often inaccurate and unreliable.
But this region is inside the adjacent block due to the overlap between them, which means substantial context information of this region can be provided to the tracing methods and leads to a better reconstruction.
Therefore, when assembling two matched neurites from adjacent blocks, we only consider neuronal compartments which are not near the boundary of the corresponding block. 

\begin{figure}[t]
	\centering
	\includegraphics[width=1\columnwidth]{./Illustrations/neuorns_fusion2.pdf}
	\caption{Two examples of correcting the over-tracing and topological discrepancy errors by our fusion algorithm. These reconstructions are shown in skeleton mode for better visualization.}
	\label{fig:overlap_discrepancy}
\end{figure}

\begin{figure}[t]
	\centering
	\includegraphics[width=1\columnwidth]{./Illustrations/trace_four_blocks2.pdf}
	\caption{One example of reconstructing neuronal populations from four adjacent blocks using our method. We can observe that, the fragmented neurites from adjacent blocks are assembled continuously and smoothly.}
	\label{fig:reconstruct_blocks}
\end{figure}

Based on the above observations, a fusion algorithm is designed to assemble the matched neurites.
Concretely, we first calculate the length of two neurites and select the longer one as the reference neurite. Here, we use $\mathcal{N_A}$ to denote the reference neurite, and $\mathcal{N_B}$ to denote another neurite.
Then, for each branch $\mathcal{H_B}$ in neurite $\mathcal{N_B}$, we search in neurite $\mathcal{N_A}$ to see if there is a branch $\mathcal{H_A}$ that overlaps with it.
%
If yes, three steps will be performed to assemble the two branches together.
Firstly, we remove the neuronal compartments in branch $\mathcal{H_A}$ and branch $\mathcal{H_B}$ that near the boundary of the corresponding block, and all branches that are connected to the removed compartments will also be removed. Here, this distance is set to be $70$ voxels.
Secondly, for each neuronal compartment in branch $\mathcal{H_B}$, if there are compartments of branch $\mathcal{H_A}$ around it, it will be removed; otherwise, it is considered to be valid. The searching radius is empirically set to be $10$ voxels.
%Thirdly, by searching the nearest compartment in branch $\mathcal{H_A}$, the modified branch $\mathcal{H_B}$ is connected to $\mathcal{H_A}$.
Thirdly, the first compartment in the modified branch $\mathcal{H_B}$ is connected to $\mathcal{H_A}$ by searching the nearest compartment in branch $\mathcal{H_A}$. 
%In this way, branches in neurite $\mathcal{N_B}$ can be merged with the neurite $\mathcal{N_A}$.
When all branches $\mathcal{H_B}$ overlapping with the neurite $\mathcal{N_A}$ are processed, we can obtain the assembled neurite $\mathcal{N_{AB}}$.
If there are remaining branches $\mathcal{H_B}$ which are not overlapped with neurite $\mathcal{N_A}$, the $\mathcal{N_{AB}}$ will be updated by assembling the remaining branches $\mathcal{H_B}$.
%After all branch $\mathcal{H_B}$ overlapping with neurite $\mathcal{N_A}$ are assembled with neurite $\mathcal{N_A}$, the remaining branches in neurite $\mathcal{N_B}$ will be connected to the assembled neurite $\mathcal{N_{AB}}$ by searching the neatest compartment in $\mathcal{N_{AB}}$.
%In this way, neurite $\mathcal{N_B}$ and neurite $\mathcal{N_A}$ can be merged.



Fig.~\ref{fig:overlap_discrepancy} shows two examples of correcting the over-tracing and topological discrepancy errors by our algorithm.
As shown in Fig.~\ref{fig:reconstruct_blocks}, the neuronal population from four adjacent blocks are successfully reconstructed from the noisy images by our UltraNPR method and the fragment neurites in adjacent blocks are assembled continuously and smoothly.
%It can be seen that the neuronal population is successfully reconstructed from the noisy images and the fragment neurites in adjacent blocks are assembled continuously and smoothly.
%The overall ``UltraNPR"" algorithm can be found in Algorithm~\ref{alg:reconstruction}.


\begin{figure}[t]
	\centering
	\includegraphics[width=\columnwidth]{./Illustrations/trace_iterations_fscore8.pdf}
	\caption{F-Score of neuron reconstruction using four neuron tracing methods APP1~\cite{Peng2011} and its variant APP2~\cite{Xiao2013}, MOST\cite{Wu2014} and NGPST~\cite{Quan2015} on the VISoR-40 test dataset at eight iterations. For each of the four neuron tracing methods, our approach progressively improves their reconstruction results.} 
	%Plots of other metrics are available in the supplementary materials.}
	\label{fig:fscore_iterations}
\end{figure}


\delete{
\subsection{Large-scale Neuronal Population Reconstruction}
\label{sec:UltraNPR}

\begin{figure*}[th]
	\centering
	\includegraphics[width=1\textwidth]{./Illustrations/framework_ultranpr.pdf}
	\caption{Diagram of our UltraNPR algorithm for neuronal population reconstruction in a large-scale brain slice.}
	\label{fig:ultra_framework}
\end{figure*}

To reconstruct neuronal populations from a large-scale image, our UltraNPR consist of three key components: a local reconstruction module, a block search module, and a neurites fusion module, as shown is~\ref{fig:ultra_framework}.
The local reconstruction module is based on our PLNPR algorithm to reconstruction neurons from low-quality image blocks. Then, to reconstruct the large-scale image block by block,  we repeatedly use the block search module to determine what the next block needs to be reconstructed. After that, the neurites fusion module is applied to obtain a complete and continuous reconstruction from adjacent blocks.

The proposed PLNPR successfully traces neurons from low-quality and megabyte-sized local image blocks. However, for a large-scale high resolution OM brain image, the image size is formidably large in a terabyte scale, which is far beyond the processing capability of existing tracing methods, especially for the memory cost.
%including our PLNPR algorithm.
%in dimension of $25397\times 18516\times 869$.
%The image used for studying large-scale neuron reconstruction is shown in Fig.~\ref{fig:brain}, a mouse brain slice with image resolution of $25397\times 18516\times 869$. 
%These terabyte-sized images are far beyond the processing capability of existing tracing methods, including our PLNPR algorithm.
%The image size is formidably large in a terabyte scale.

To reconstruct neuronal populations from a large-scale image, the basic idea of our UltraNPR algorithm is to reconstruct neurons block by block.
The image used in this work is shown in Fig.~\ref{fig:brain}, a mouse brain slice with image resolution of $25397\times 18516\times 869$.
%
As shown in Fig~\ref{fig:ultra_framework}, our UltraNPR algorithm consists of three key components.
%: soma detection, 
%
First of all, the large-scale image is divided into 3D blocks of the same size. 
Considering the raw image size and the computational efficiency, the block size is set to be $1120\times 2048\times 869$. 
In addition, we introduce $300$ voxels overlap between adjacent blocks to avoid false continuation and increase the robustness of the reconstruction.
%avoid detection errors and tracing errors in soma detection and neuron reconstruction separately.
To obtain a complete and continuous reconstruction from blocks, UltraNPR consists of two steps: soma detection and neuronal population reconstruction.

\subsubsection{Initial Soma Detection}
\label{sec:soma}

To detect somas from the large-scale image efficiently, we apply soma detection algorithm~\cite{Quan2013} on each block separately.
However, due to the overlap between blocks, somas in the overlapped area would be detected repeatedly.
%To remove these duplicate detection results, we first map all somas to a unified spatial coordinate system of the brain slice.
To tackle this over-detection error, we map all detected somas from local blocks to the coordinate system of the original image and merge the overlapped somas by averaging their position and radius.
%the position and radius of the merged soma are the average of the position and radius of the overlapped somas, respectively.
%Fig.~\ref{fig:detected_soma} shows the detected somas of the brain image.
The soma detection results from the large-scale image are visualized in Fig.~{4} of the supplementary file.
These somas are also used to define the initial blocks for neuron reconstruction in the next step.


%\subsubsection{Large-scale Neuronal Population Reconstruction}
\subsubsection{Block Search Policy and Local Reconstruction}
\label{sec:trace}

The neuronal population reconstruction $\mathbf{R}$ in a large-scale brain image can be modeled as the joint performance of the reconstruction in each block $\mathbf{B}$.
Given an image $\mathbf{I}$, performing a tracing method $T$ to reconstruct neuronal populations block by block can be formulated as follows
\begin{equation}
\mathbf{R} = T(\mathbf{I}) = T(\mathbf{B}_1, \mathbf{B}_2,\cdots,\mathbf{B}_N).
\end{equation}
%
Since neurons would be split into fragmented neurites when dividing raw image into blocks, neurites from adjacent blocks need to be maximally connected, and the connection should be as continuous and smooth as possible.
We use an intuitive approach to solve this problem, by repetitively searching the next block according to the neurons in already-reconstructed blocks. This process can be factorized as follows
\begin{equation}
\begin{aligned}
&\mathbf{R}_1 =  T(\mathbf{B}_1), \\
&\mathbf{R}_2 =  T(\mathbf{B}_2 | \mathbf{R}_1), \\
&\mathbf{R}_3 =  T(\mathbf{B}_3 | \mathbf{R}_1, \mathbf{R}_2),\\
&\cdots\\
&\mathbf{R} =  T(\mathbf{B}_N | \mathbf{R}_1, \mathbf{R}_2, \cdots, \mathbf{R}_{N-1}).\\
\end{aligned}
\end{equation}

As somas are where signals from the dendrites are joined and pass on, the blocks containing somas are the most probable locations to start the tracing.
UltraNPR uses the PLNPR method to reconstruct neurons from the blocks that contain somas at first.
%Typically, the neuronal compartments closed to the block boundary indicate the continuity of the structure of neurons. Therefore, we detect the terminal tips from these boundary compartments.
The neuron reconstruction produced by PLNPR is represented by a series of neuronal compartments. The neuronal compartments which are close to the block boundary typically indicate the continuity of the structure of neurons. We call these compartments as terminal tips and use them as the potential continuous signals for searching the next blocks to grow the neuronal structure from each initial block.
%To grow the neuronal structure from each initial block, terminal tips are used as the potential continuous signals for searching the next blocks.
%Typically, the neuronal compartments closed to the block boundary indicate the continuity of the structure of neurons. Therefore, we detect the terminal tips from these boundary compartments.
Then, neurons in the next blocks are reconstructed by PLNPR based on the terminal tips provided by adjacent blocks that have already been reconstructed.
By iteratively searching the next blocks, UltraNPR can reconstruct neuronal populations more and more completely.
%
However, due to the dense distribution of neurons in the brain image, fragmented neurites belonging to different neurons could be in the same block, which means some blocks may be repeatedly reconstructed starting from different initial blocks.
To more efficiently explore the structure of neuronal populations, for each searching direction of an initial block, the iterative searching process will be terminated when the next block contains somas or no new terminal tips could be detected.



%%%%%%%%%%%%%%%%%%%%%%%%%%%%%%%%%%%%%%%%%%%%%%
\subsubsection{Neurites Fusion from Adjacent Blocks}
After reconstructing neurons from all blocks, UltraNPR analyzes the reconstructed neurites to obtain a complete neuronal population reconstruction.
%between adjacent blocks, and applies the following fusion algorithm to assemble them.
%
Since fragment neurites in the overlapping area may belong to different neurons, we first match them by comparing the overlapping region between every two neurites from adjacent blocks.
%Since fragment neurites of different neurons may be in the overlapping area between two adjacent blocks, we first map these neurites by comparing the overlapping region between neurites.
Then, the matched neurites are assembled to get a continuous and smooth reconstruction. 


However, simply assembling them would cause over-tracing error and topological discrepancy, as shown in Fig.~\ref{fig:overlap_discrepancy}.
%
The over-tracing error is caused by the overlap between blocks when the respective neurites from these blocks are assembled.
%so that the overlapped neuronal compartments need to be assembled during the fusion process.
The topological discrepancy is mainly caused by the lack of context information for tracing methods, which makes the reconstruction near the block boundary often inaccurate and unreliable.
At the same time, this region is inside the adjacent block due to the overlap between them, which means substantial context information of this region can be provided to the tracing methods and leads to a better reconstruction.
Therefore, when assembling two matched neurites from adjacent blocks, we only consider neuronal compartments which are not near the boundary of the corresponding block. 
%Here, this distance is empirically set to be $70$ voxels.
%``neurite A"``neurite B"

\begin{figure}[t]
	\centering
	\includegraphics[width=1\columnwidth]{./Illustrations/neuorns_fusion2.pdf}
	\caption{Two examples of correcting the over-tracing and topological discrepancy errors by our fusion algorithm. These reconstructions are shown in skeleton mode for better visualization.}
	\label{fig:overlap_discrepancy}
\end{figure}

\begin{figure}[t]
	\centering
	\includegraphics[width=1\columnwidth]{./Illustrations/trace_four_blocks2.pdf}
	\caption{One example of reconstructing neuronal populations from four adjacent blocks using our method. We can observe that, the fragmented neurites from adjacent blocks are assembled continuously and smoothly.}
	\label{fig:reconstruct_blocks}
\end{figure}

Based on the above observations, we design a new fusion algorithm to assemble the matched neurites.
Concretely, we first calculate the length of the two neurites and select the longer one as the reference neurite. Here, we use $\mathcal{N_A}$ to denote the reference neurite, and $\mathcal{N_B}$ to denote another neurite.
Then, for each branch $\mathcal{H_B}$ in neurite $\mathcal{N_B}$, we search in neurite $\mathcal{N_A}$ to see if there is a branch $\mathcal{H_A}$ that overlaps with it.
%
If yes, three steps will be performed to assemble the two branches together.
Firstly, we remove the neuronal compartments in branch $\mathcal{H_A}$ and branch $\mathcal{H_B}$ that near the boundary of the corresponding block, and all branches that are connected to the removed compartments will also be removed. Here, this distance is set to be $70$ voxels.
Secondly, for each neuronal compartment in branch $\mathcal{H_B}$, if there are compartments of branch $\mathcal{H_A}$ around it, it will be removed; otherwise, it is considered to be valid. The searching radius is empirically set to be $10$ voxels.
%Thirdly, by searching the nearest compartment in branch $\mathcal{H_A}$, the modified branch $\mathcal{H_B}$ is connected to $\mathcal{H_A}$.
Thirdly, the first compartment in the modified branch $\mathcal{H_B}$ is connected to $\mathcal{H_A}$ by searching the nearest compartment in branch $\mathcal{H_A}$. 
%In this way, branches in neurite $\mathcal{N_B}$ can be merged with the neurite $\mathcal{N_A}$.
When all branches $\mathcal{H_B}$ overlapping with the neurite $\mathcal{N_A}$ are processed, we can obtain the assembled neurite $\mathcal{N_{AB}}$.
If there are remaining branches $\mathcal{H_B}$ which are not overlapped with neurite $\mathcal{N_A}$, the $\mathcal{N_{AB}}$ will be updated by assembling the remaining branches $\mathcal{H_B}$.
%After all branch $\mathcal{H_B}$ overlapping with neurite $\mathcal{N_A}$ are assembled with neurite $\mathcal{N_A}$, the remaining branches in neurite $\mathcal{N_B}$ will be connected to the assembled neurite $\mathcal{N_{AB}}$ by searching the neatest compartment in $\mathcal{N_{AB}}$.
%In this way, neurite $\mathcal{N_B}$ and neurite $\mathcal{N_A}$ can be merged.



Fig.~\ref{fig:overlap_discrepancy} shows two examples of correcting the over-tracing and topological discrepancy errors by our algorithm.
As shown in Fig.~\ref{fig:reconstruct_blocks}, the neuronal population from four adjacent blocks are successfully reconstructed from the noisy images by our UltraNPR method and the fragment neurites in adjacent blocks are assembled continuously and smoothly.
%It can be seen that the neuronal population is successfully reconstructed from the noisy images and the fragment neurites in adjacent blocks are assembled continuously and smoothly.
%The overall ``UltraNPR"" algorithm can be found in Algorithm~\ref{alg:reconstruction}.


\begin{figure}[t]
	\centering
	\includegraphics[width=\columnwidth]{./Illustrations/trace_iterations_fscore8.pdf}
	\caption{F-Score of neuron reconstruction using four neuron tracing methods APP1~\cite{Peng2011} and its variant APP2~\cite{Xiao2013}, MOST\cite{Wu2014} and NGPST~\cite{Quan2015} on the VISoR-40 test dataset at eight iterations. For each of the four neuron tracing methods, our approach progressively improves their reconstruction results.} 
	%Plots of other metrics are available in the supplementary materials.}
	\label{fig:fscore_iterations}
\end{figure}

}
\section{Experiments and Results}
\label{sec:experiments}

We evaluate our neuron population reconstruction approach in an ultra-scale OM image slice of a mouse brain in dimension of $25397\times 18516\times 869$ ($761$ GB), as Fig.~\ref{fig:brain} shows.
%
The image was captured by the VISoR imaging system~\cite{Wang2019} at a physical resolution of $0.5 \times0.5 \times 0.5$ \SI{}{\micro\metre}$^3$ per voxel. 
%
The image intensity is in 16-bit dynamic range, which preserves sufficient signal details.
In order to evaluate our PLNPR method for neuronal reconstruction in local blocks, we first conduct extensive experiments on the VISoR-40 dataset which we build and the BigNeuron dataset~\cite{peng2015}. 
%
Then we test our UltraNPR algorithm for neuronal population reconstruction in a slice of the mouse brain image.

\subsection{Evaluation of PLNPR on VISoR-40 Dataset}
\label{sec:exp_PLNPR_VISoR}

\subsubsection{VISoR-40 Dataset}
Though many neuron tracing techniques have been proposed, no dataset of OM images has been built for dense neuronal population reconstruction.
We construct a neuron image dataset ``VISoR-40'' (available at \url{https://braindata.bitahub.com/Neuronal_population_reconstruction.html}) for evaluation. 
The VISoR-40 dataset consists of 40 OM image blocks cropped from the mouse brain image. The dimension of the blocks ranges from $419 \times1197 \times 224$ to $869 \times1853 \times 575$.
%
We randomly select $32$ blocks for progressively training the segmentation network without manual annotation in our PLNPR.
%
The remaining 8 blocks with manual annotations are used as the test data.
Each image block for test was first labeled manually and independently by two experts. Then, by cross-checking their results, their agreed annotations were approved by another expert to generate the final ground truth.

\subsubsection{Experimental Settings and Evaluation Metrics}


Pytorch is adopted to implement the DSN model. At each iteration of the progressive learning, the network is trained from scratch with weights initialized from a Gaussian distribution with zero-mean and variance of $ 0.01 $. The optimization is realized with the stochastic gradient descent algorithm with the Adam update rule (batch size of 1, weight decay of $ 0.0005 $, momentum of $ 0.9 $). The base learning rate is set to $ 0.001 $ and descended with the ``poly" learning rate policy (power of $ 0.9 $ and the maximum iteration number of $ 24000 $). 
%The cube size is set as $160\times 160\times 160$ considering the GPU memory limitation.

To quantitatively evaluate our method, four commonly used metrics defined in NGPST~\cite{Quan2015}, including precision, recall, F-Score, and Jaccard, are computed to measure the fidelity between the reconstruction results and the ground truth. 
Their definitions are defined as follows:
\begin{flalign}
&Precision(R,G)= \frac{\vert R \cap G\vert}{\vert R\vert} = \frac{\vert TP\vert}{\vert R\vert}, & \\
&Recall(R,G) = \frac{\vert R\cap G\vert}{\vert G\vert} = \frac{\vert TP\vert}{\vert G\vert}, & \\
&F{-}Score(R,G) = 2 \cdot{\frac{Precision \times Recall}{Precision + Recall}},&\\
%&F{-}Score(R,G)= \frac{2\vert R \cap G\vert}{\vert R\vert + \vert G\vert} = \frac{2\vert TP\vert}{\vert R\vert + \vert G\vert}, & \\
&Jaccard(R,G)= \frac{\vert R\cap G\vert}{\vert R\cup G\vert} = \frac{\vert TP\vert}{\vert R\cup G\vert}, &
\label{equ: metrics}
\end{flalign}
%
where $R$ denotes the set of points on the reconstructed neurons, $G$ denotes the set of neuron points in the ground truth and $TP$ denotes the set of true positive points, $|\cdot|$ denotes the number of points in a set.
%We follow the rule of determining true positive points described in~\cite{Quan2015}. 
The four metrics are first computed on each individual neuronal tree according to the manually labeled skeleton, and then averaged in a neuronal population weighted by the total length of the neuronal processes of each neuron, the same as \cite{Quan2015}.



\subsubsection{Progressive Learning}

The key idea of PLNPR is to progressively improve the performance of neuron reconstruction by making the neuron segmentation network and the conventional tracing method complementary and synergistic without using any manual annotations.
In order to demonstrate the performance improvement, four widely-used tracing methods, including APP1~\cite{Peng2011}, APP2~\cite{Xiao2013}, MOST~\cite{Wu2014} and NGPST~\cite{Quan2015}, are tested as the neuron tracing module in our framework. 
We use their implementations in the software Vaa3D~\cite{Peng2014}. 
%
Eight iterations are tested on our VISoR-40 dataset, and the improvement of neuronal population reconstruction is shown in Fig.~\ref{fig:ablation_study_plnpr}~(a).
We only show the F-Score which is widely used to reflect the overall performance of neuron reconstruction.
%
Moreover, the neuron reconstruction results on a test block at different iterations are shown in the Fig.~\ref{fig:trace_iterations}.
%
More qualitative and quantitative results are reported in the supplementary materials.
The results show that our progressive learning strategy effectively facilitates conventional tracing methods to reconstruct more complete neuronal populations.
In addition, the performance improvement gets stable after five iterations of the progressive learning for all the tested tracing methods. 

\begin{figure*}[t]
	\centering
	\subfloat[]{
		\begin{minipage}[b]{0.33\linewidth}
			\centering
			\includegraphics[height=3.8cm]{./Illustrations/PLNPR_fscore.pdf}
		\end{minipage}}
	\subfloat[]{
		\begin{minipage}[b]{0.33\linewidth}
			\centering
			\includegraphics[height=3.8cm]{./Illustrations/PLNPR_networks.pdf}
		\end{minipage}}
	\subfloat[]{
		\begin{minipage}[b]{0.33\linewidth}
			\centering
			\includegraphics[height=3.8cm]{./Illustrations/PLNPR_alpha.pdf}
		\end{minipage}}
	\caption{ Comparison of different parameters in our PLNPR framework on the VISoR-40 dataset.
		(a) Comparisons of neuronal population reconstruction performance at different iterations using four neuron tracing methods. 
		(b) F-Score of neuron reconstruction at five iterations using three deep segmentation networks. 
		Combining any base tracer and any one of the three neuron segmentation networks, our approach progressively improves the reconstruction performance.
		(c) Neuron reconstruction performance with different $\alpha$ in Eq.~\eqref{equ: enhance} for image enhancement.} %From left to right, the value of $\alpha$ increases from $0$ to $1$ by a step of $0.1$.
	
	\label{fig:ablation_study_plnpr}
\end{figure*}


\begin{figure}[t]
	\centering
	\includegraphics[width=1\columnwidth]{./Illustrations/trace_iterations3.pdf}
	\caption{Neuronal population reconstruction results of a test block at different iterations (top to bottom) using four neuron tracing methods.} 
	% APP1~\cite{Peng2011}, APP2~\cite{Xiao2013}, MOST\cite{Wu2014} and NGPST~\cite{Quan2015}.}
	\label{fig:trace_iterations}
\end{figure}

\delete{
\begin{figure}[t]
	\centering
	\includegraphics[width=0.8\columnwidth]{./Illustrations/trace_networks_fscore11.pdf}
	\caption{F-Score of neuron reconstruction on the VISoR-40 testing dataset at five iterations. Combining any one of the three neuron segmentation networks, our approach progressively improves the reconstruction performance.}
	\label{fig:fscore_DNNs}
\end{figure}
}


\subsubsection{Neuron Segmentation Network}

To further verify the effectiveness and robustness of our progressive learning strategy, we test three commonly-used deep segmentation networks, including 3D DSN~\cite{Dou2017}, 3D U-Net~\cite{Cicek2016} and a 3D version of HRNet~\cite{Sun2019}, for generating the neuron probability map.
Five iterations are tested on our VISoR-40 dataset, and the F-Score improvement of reconstruction results is shown in Fig.~\ref{fig:ablation_study_plnpr}~(b). 
%
It can be seen that our PLNPR algorithm effectively improves the neuron reconstruction performance by combining any one of the three neuron segmentation networks.
Consequently, the segmentation network and traditional tracing method can complement and promote each other, leading to more complete neuron reconstruction.


\subsubsection{Enhancement Parameter} 

In order to explore the influence of parameter $\alpha$ in Eq.~\eqref{equ: enhance} for image enhancement, we adopt different values for $\alpha$, and the results are shown in Fig.~\ref{fig:ablation_study_plnpr}~(c).
$\alpha=0$ means that the raw image block is directly used as input for the tracing module.
$\alpha=1$ means that only the probability maps are used as input for neuron tracing. 
It indicates that the performance is improved by combing the probability map with the raw image signal, mainly because that the probability map reflects the long-range trajectory structures while the original image signal carries more details of subtle neurites.
%Second, it can be observed our system is very robust to the parameter $\alpha$. 
%Hence, the chosen of the parameter $\alpha$ is not very demanding and the reconstruction performance can achieve significant improvement with any $\alpha >0$. 
In this paper, we empirically select $\alpha=0.1$ to reduce the influence of false positive predictions in probability maps due to the limited performance of the DNN model trained by pseudo labels and increase the robustness of the whole framework.

\delete{
\begin{figure}[t]
	\centering
	\includegraphics[width=0.8\columnwidth]{./Illustrations/weight_paprameter7.pdf}
	\caption{Neuron reconstruction performance with different $\alpha$ in Eq.~\eqref{equ: enhance} for image enhancement on the VISoR-40 test dataset. From left to right, the value of $\alpha$ increases from $0$ to $1$ by a step of $0.1$.  }
	\label{fig:weight_paprameter}
\end{figure}
}


\subsubsection{Comparison with Tracing Methods}

\begin{figure*}[t]
	\centering
	\includegraphics[width=1\textwidth]{./Illustrations/iteration3.pdf}
	\caption{Comparison of neuronal population reconstruction results of three image blocks. %using neuron reconstruction methods FMST~\cite{Yang2019}, APP1~\cite{Peng2011}, APP2~\cite{Xiao2013}, SmartTracing~\cite{Chen2015}, MOST~\cite{Wu2014}, NGPST~\cite{Quan2015} and our PLNPR on three test images from the VISoR-40 dataset.
	%Each row shows the reconstruction results generated by different methods for a test image. The first column shows the raw images, while the last column shows the ground truth (GT). Each of the remaining columns shows the reconstruction result using the corresponding tracing method. 
	Our PLNPR method reconstructs more complete and accurate neurons compared to other methods. Separated neurons are shown in different colors.
	}
	\label{fig:compare_VISoR}
\end{figure*}

\begin{table*}[t]
	\centering
	\caption{Performance comparison with different methods for neuronal population reconstruction on the VISoR-40 test dataset given raw images and the same enhanced images as input.}
	\label{table:compare_VISoR}
	\begin{tabular}{lcccc|cccc}
		\hline
		\multirow{2}{*}{Method} & \multicolumn{4}{c|}{Raw Images}  & \multicolumn{4}{c}{Enhanced Images}\\ 
		\cline{2-5} \cline{6-9}
		& Precision & Recall & F-Score & Jaccard & Precision & Recall & F-Score & Jaccard\\ 
		\hline
		\multicolumn{1}{l}{APP2~\cite{Xiao2013}} & \textbf{0.980} & 0.115 & 0.191 & 0.119 & 0.973 & 0.447 & 0.604 & 0.480\\
		\multicolumn{1}{l}{SmartTracing~\cite{Chen2015}} & 0.877 & 0.291 & 0.400 & 0.242 & 0.917 & 0.622 & 0.701 & 0.480\\
		\multicolumn{1}{l}{TReMAP~\cite{Zhou2016}} & 0.917 & 0.199 & 0.326 & 0.203 & 0.906 & 0.421 & 0.573 & 0.429\\
		\multicolumn{1}{l}{MOST~\cite{Wu2014} } & 0.969 & 0.284& 0.434& 0.316 & 0.953 & 0.597 & 0.725 & 0.655\\
		\multicolumn{1}{l}{APP1~\cite{Peng2011}} & 0.935 & 0.201 & 0.328 & 0.205 & 0.930 & 0.516 & 0.658 & 0.507\\
		\multicolumn{1}{l}{FMST~\cite{Yang2019}} & 0.884 & 0.208 & 0.335 &  0.211 & 0.916 & 0.435 & 0.588 &  0.439\\
		\multicolumn{1}{l}{NGPST~\cite{Quan2015}} & 0.978 & 0.603 & 0.741 & 0.623 & 0.971 & \textbf{0.829} & \textbf{0.893} & \textbf{0.833}\\
		\hline
	\end{tabular}
\end{table*}


%<18-AAAI-Adaptive Graph Convolutional Neural Networks>
To prove the effectiveness of our method on neuronal population reconstruction, we compare it with seven widely used neuron tracing methods, including APP1~\cite{Peng2011},  APP2~\cite{Xiao2013}, MOST~\cite{Wu2014}, SmartTracing~\cite{Chen2015}, NGPST~\cite{Quan2015}, TReMAP~\cite{Zhou2016} and  FMST~\cite{Yang2019}.
The parameters of these tracing methods are manually adjusted for each image block to get the optimal performance in our experiments.
%
Fig.~\ref{fig:compare_VISoR} shows the neuronal populations reconstructed from three test image blocks.
We utilize NGPST with 3D DSN enhancement as ``Ours''. Note that the segmentation network in our approach is trained progressively on the VISoR-40 dataset in the training stage. We use the trained model directly at the test stage for evaluation. 
Compared with other methods, our PLNPR is superior in both sparse and dense neurons.
%
Table~\ref{table:compare_VISoR} compares the quantitative results of different methods with regard to the four metrics including precision, recall, F-Score, and Jaccard.
%
It shows that our PLNPR makes a significant improvement on the overall performance compared with other methods.
Though APP2 achieves the highest precision, the reconstructed neurons are significantly sparser than others.
%
Conventional tracing methods~\cite{Peng2011, Xiao2013, Wu2014, Zhou2016} and learning-based methods~\cite{Chen2015, Yang2019} tend to extract the main trunk of neurons, while missing a large portion of subtle neurites. 
Therefore, these methods have very high precision but significantly lower recall.
Although NGPST~\cite{Quan2015} achieves better performance of neuronal population reconstruction compared with other single-neuron tracing methods, it still remains difficult to extract subtle neuron voxels for NGPST by using hand-crafted features.
%
In comparison, our method benefits from the progressively trained segmentation network, and reconstructs more complete neurons from challenging blocks, even there exhibit noises, low contrast, and blending of fluorescence in the blocks.
%\textsc{\textsc{}}
Furthermore,  the quantitative results of each tracing method when given raw images and the same enhanced images which are produced by our PLNPR method are compared in Table~\ref{table:compare_VISoR}.
The results show that each method is promoted to achieve significantly higher overall performance and reconstruct more complete neuronal populations when using the same enhanced images as input.
%


\subsection{Evaluation of PLNPR on BigNeuron Dataset}
\label{sec:exp_PLNPR_BigNeuron}

\subsubsection{BigNeuron Dataset}

To validate our PLNPR method on single neuron reconstruction, we employ the BigNeuron~\cite{peng2015} dataset.
% which is a well-known community-derived neuron dataset. 
This dataset consists of about $20,000$ 3D OM images in total, acquired from a variety of species and optical imaging systems by different institutes.
%Some images have the corresponding manual annotations for evaluation.  
Unlike our VISoR-40 dataset which is built for the evaluation of neuronal population reconstruction, each block in the BigNeuron dataset only contains a single neuron or fragmented neurites.
% which are appropriate for single neuron reconstruction.
Following \cite{Li2017}, we select the same 68 images that are from a variety of species to evaluate the performance of dense neurite reconstruction.
Manual reconstruction by experts is associated with each image. 
51 images are used for network training in \cite{Li2017} and the remaining 17 images are used for evaluation.
Note that we do not use the manual annotations in our PLNPR in training the deep neural network. 


\subsubsection{Experimental Settings}
 
 
To evaluate our PLNPR on the single neuron reconstruction, we progressively train the DSN model on the BigNeuron dataset using pseudo labels generated by a conventional tracing method, such as NGPST~\cite{Quan2015} and APP2~\cite{Xiao2013}, instead of the provided manual annotations.
%
The learning rate was initialized as $\num{1e-4}$ and decayed using the ``poly" learning rate policy with power of $0.9$. The maximum iteration number is set to $ 24000 $. 
We cropped image patches of size $160\times 160\times 8$ as input to the segmentation network since the axial dimensions are usually much lower in the images of the BigNeuron dataset than our VISoR-40 dataset.
Data augmentation by transposing the three dimensions of each training image is also performed. 


\subsubsection{Comparison on BigNeuron Dataset}

\begin{figure*}[th]
	\centering
	\includegraphics[width=0.9\textwidth]{./Illustrations/BigNeuron_comparison_v2.pdf}
	\caption{Comparison of single neuron reconstruction results on two test images from the BigNeuron dataset.
		% using MOST~\cite{Wu2014}, FMST~\cite{Yang2019}, APP2~\cite{Xiao2013}, TReMAP~\cite{Zhou2016}, NGPST~\cite{Quan2015}, SmartTracing~\cite{Chen2015}, Li2017~\cite{Li2017} and our PLNPR on two testing images from the BigNeuron dataset.
	The reconstructed neurites are shown in blue and the corresponding ground truth (GT) are shown in red.
	Our method reconstructs more complete and accurate neurons compared to other methods.
	}
	\label{fig:compare_BigNeuron}
\end{figure*}

\begin{table*}[h]
	\centering
	\makeatletter\def\@captype{table}\makeatother
	\caption{Performance comparison for single neuron reconstruction on the BigNeuron test dataset.}
	\label{table:compare_BigNeuron}
	\begin{tabular}{lcccccccc}
		\toprule
		Method & Precision & Recall & F-Score & Jaccard & ESA & DSA & PDS\\
		\midrule
		MOST~\cite{Wu2014} & 0.629 & 0.508 & 0.541 & 0.593 & 31.730 & 38.211 & 0.633\\
		TReMAP~\cite{Zhou2016} & 0.771 & 0.508 & 0.578 & 0.525 & 11.269 & 17.941 & 0.539\\
		FMST~\cite{Yang2019} & 0.575 & 0.658 & 0.591 & 0.532 & 17.878 & 23.459 & 0.558\\
		neuTube~\cite{Feng2015} & 0.742&0.738&0.710&0.648&12.741&21.379&0.434&\\
		SmartTracing~\cite{Chen2015} & 0.701 & 0.722 & 0.613 & 0.465 & 10.430 & 13.430 & 0.547\\
		APP2~\cite{Xiao2013} & 0.875 & 	0.582 & 0.681 & 0.667 & 6.063 & 9.079 & 0.496\\
		NGPST~\cite{Quan2015} & 0.710 & 0.757 & 0.728 & 0.592 & 8.830 & 17.489 & 0.463\\
		MEIT~\cite{Wang2018} & 0.689  & 0.729 & 0.702 & 0.396 & 11.635 &15.772 & 0.544 \\
		UltraTracer~\cite{Peng2017} & 0.796 & 0.685 & 0.705 & 0.682 & 10.156 & 17.355 & 0.426 \\
		Li2017~\cite{Li2017} & - & - & - & - & 4.917 & 7.972 &0.461 \\
		\midrule
		PLNPR-APP2 & \textbf{0.877} & 0.616 & 0.713 & \textbf{0.706} & 5.020 & \textbf{7.753} & 0.492\\
		PLNPR-NGPST &0.783&\textbf{0.789}&\textbf{0.778}&0.691&\textbf{4.714}&9.458 & \textbf{0.428}\\
		%PLNPR-NGPST (VISoR-40+BigNeuron) & 0.769 & 0.777 & 0.768 & 0.683 & 4.694 & 8.082 & 0.445\\
		\bottomrule
	\end{tabular}
\end{table*}

\begin{figure*}[t]
	\centering
	\includegraphics[width=\textwidth]{./Illustrations/comparison_ultranpr2.pdf}
	\caption{Reconstruction results of dense neuronal populations from four adjacent large-scale blocks using UltraTracer~\cite{Peng2017}, MEIT~\cite{Wang2018} and our UltraNPR. The second row shows close-up views for a local region with dense neurites. Our method reconstructs more complete and distinguishable neurons. 
	}
	\label{fig:reconstruct_blocks}
\end{figure*}

\begin{figure}[t]
	\centering
	\includegraphics[width=1\columnwidth]{./Illustrations/brain_slice.pdf}
	\caption{The reconstruction result of neuronal populations in a large-scale 3D mouse brain slice using our UltraNPR method.}
	\label{fig:reconstruct_brain}
\end{figure}

On the BigNeuron dataset, we compare with ten widely used tracing methods to validate the effectiveness of our proposed method.
They are MOST~\cite{Wu2014}, TReMAP~\cite{Zhou2016}, FMST~\cite{Yang2019}, neuTube~\cite{Feng2015}, SmartTracing~\cite{Chen2015}, APP2~\cite{Xiao2013}, NGPST~\cite{Quan2015}, Li2017~\cite{Li2017},   MEIT~\cite{Wang2018} and UltraTracer~\cite{Peng2017} respectively.
%
The four metrics, including precision, recall, F-score, and Jaccard, are used for comparison for most methods.
However, the implementation of most learning-based tracing methods, such as~\cite{Li2017}, are not available.
In order to compare with~\cite{Li2017}, we only compare the three evaluation metrics reported in \cite{Li2017} on the same test data.
%
The three metrics defined in~\cite{Peng2010a} include the entire structure average (ESA), different structure average (DSA) and percentage of different structures (PDS).
%

We evaluate the results obtained from different methods on the BigNeuron dataset in Table~\ref{table:compare_BigNeuron}.
``PLNPR-NGPST" means that we progressively train the DSN model using pseudo labels generated by NGPST~\cite{Quan2015}. In addition, we test our PLNPR by progressively learning with APP2~\cite{Xiao2013} (``PLNPR-APP2") which is same as the base tracer used in Li2017~\cite{Li2017}.
%
The weighted averages of the ESA, DSA and PDS are calculated by setting the of each test block proportional to the neuron length identified in the corresponding manual annotation.
For these three scores, larger values indicate higher discrepancy between the tracing results and the manual reconstruction.
%
Fig.~\ref{fig:compare_BigNeuron} shows the reconstructed neurons from two test images using different methods.
%
From Table~\ref{table:compare_BigNeuron}, we can see that our method outperforms other methods on the BigNeuron dataset.
%Though APP2~\cite{Xiao2013} achieves the highest precision, a large portion of subtle neurites are missing, as shown in Fig.~\ref{fig:compare_BigNeuron}.
%
Comparing with \cite{Li2017}, our PLNPR achieves comparable performance.
%
However, our method does not require any manual annotations to train the deep segmentation network.
With ever-increasing number of unlabeled neuron datasets are collected, our method could utilize them to further improve the performance of neuron reconstruction.




\subsection{Evaluation of UltraNPR on a Mouse Brain Slice}
\label{sec:exp_UltraNPR}

%\subsubsection{Experimental Settings}
To reconstruct the dense neuron population from an ultra-scale image, we divide the entire image into blocks in size of $1120\times 2048\times 869$ considering the memory and computational efficiency of our PLNPR.
%
The overlap between adjacent blocks is set to be $300$ voxels along each dimension. 
The $\delta_{ovlp}$ is set to be 10 voxels.
The $\delta_{bound}$ is set to be 70 voxels and the $\delta_{pt}$ is set to be 10 voxels for neurite fusion in Sec.~\ref{sec:fusion}.
%
It took our UltraNPR about 34 hours for deep image segmentation, 1 hour for neuron reconstruction in blocks, and 10 hours for neurite fusion to reconstruct the dense neuron population from the entire image on a cluster computer with $64$ GB of working memory and 20 NVIDIA 1080Ti GPUs.


Since it is infeasible to manually annotate the dense nueron population in an ultra-scale image, quantitative evaluation of the reconstruction performance is not supported.
%
However, we select four adjacent large-scale image blocks to qualitatively compare our UltraNPR with two state-of-the-art methods, UltraTracer~\cite{Peng2017} and MEIT~\cite{Wang2018}, for large-scale neuron reconstruction.
In addition, to verify the effectiveness of PLNPR for UltraNPR, we compare with NGPST~\cite{Quan2015} which uses raw image blocks as input to reconstruct neurons and ultlizes the block propagation and neurite fusion algorithms which are proposed in our UltraNPR to achieve the large-scale neuonal population reconstruction.
%

\begin{figure}[t]
	\centering
	\includegraphics[width=\columnwidth]{./Illustrations/single_neurons4.pdf}
	\caption{Single neurons selected from the reconstructed neuronal populations in a mouse brain slice using our UltraNPR method.}
	\label{fig:single_neurons}
\end{figure}

Fig.~\ref{fig:reconstruct_blocks} shows the neuronal populations reconstructed results.
%
MEIT~\cite{Wang2018} is designed for tracing single neuron, without separating individual neurons. 
Moreover, it fails to reconstruct the subtle dendrites due to the noises and low contrast in our challenging image.
%\xj{Another drawback of MEIT is that many parameters have to be carefully tuned to obtain satisfied results. More results under different parameters using MEIT are shown in our supplementary files. }
% 
UltraTracer~\cite{Peng2017} achieves better performance of neuronal population reconstruction from the large-scale image. 
However, for the local regions with low signal-noise-ratio, it fails to separate individual neurons and trace complete dendrites in a dense neuron population.
Although NGPST~\cite{Quan2015} can reconstruct separated individual neurons successfully, it still remains difficult to reconsruct complete neuronal populations from the challenging image for NGPST using hand-crafted features.
In comparison, thanks to the signal enhancement by our deep network and block propagation designed for dense neurites, our UltraNPR is more robust to reconstruct a more complete neuronal population from the low-quality image while individual neurons are continuously and smoothly traced.


As shown in Fig.~\ref{fig:reconstruct_brain}, a neuronal population which consists of $5348$ neurons is successfully reconstructed by our UltraNPR from the mouse brain slice. Several neurons selected from the reconstructed neuronal population are visualized in Fig.~\ref{fig:single_neurons}. 
We believe that these large-scale reconstructions provide detailed neuronal structures and will effectively support further neuronal morphology analysis in the whole brain. 
In summary, without any parameter-tuning and human interaction, our UltraNPR is capable of reconstructing dense neuronal populations from ultra-scale noisy OM images. 


\section{Conclusion}
\label{sec:conclusion}
In this work, we propose PLNPR, an \md{unsupervised} progressive learning framework for neuronal population reconstruction from noisy and low-quality OM image blocks.
Without using any manual annotations, we take advantage of neuron tracing techniques and deep segmentation networks, and make them mutually complement and promote each other progressively.
We extensively validate the proposed PLNPR for neuron reconstruction and the results demonstrate the effectiveness and superiority of our method.
%This framework can progressively train a deep segmentation network for extracting neuron voxels without using any manual an5421`notations. 
%We extensively validate the proposed 3D DSN on two distinct applications and the results demonstrate the effectiveness and generalization of the proposed network.
Furthermore, we introduce UltraNPR, a reconstruction algorithm that makes possible the neuronal population reconstruction from large-scale OM brain images.
We also construct a new neuron dataset ``VISoR-40'' which consists of $ 40 $ OM image blocks for the evaluation of neuronal population reconstruction.
This dataset will be published to facilitate further work on brain research, including but not limited to neuron counting, neuron reconstruction and neuron morphology analysis, and so on.

\bibliographystyle{model2-names.bst}\biboptions{authoryear}
\bibliography{NeuroSegRef}

\end{document}

